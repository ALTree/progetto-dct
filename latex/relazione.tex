\documentclass[11pt,a4paper]{scrartcl}
\usepackage[utf8]{inputenc}
\usepackage[italian]{babel}

\usepackage{amsmath, amsfonts, amssymb}
\usepackage{graphicx, booktabs}
\usepackage{minted}

\usepackage{multirow}
\usepackage{parskip}

\author{L. De Sano, A. Donizetti, M. Scotti}
\title{La trasformata DCT applicata alle immagini}
\date{Maggio 2014}

\begin{document}
\maketitle
\begin{abstract}
Descriviamo l'implementazione di una serie di test riguardanti l'applicazione di una trasformata DCT2 ad immagini bitmap \textit{grayscale}, il taglio di alcune delle frequenze della trasformata, l'applicazione della IDCT2 alla matrice risultante, e il confronto con l'originale dell'immagine BMP ottenuta dopo il taglio.
\end{abstract}

\section*{Implementazione}

In questa sezione diamo una descrizione dell'implementazione dei test richiesti dalla consegna. Le parti di lettura file BMP, calcolo DCT2 e IDCT2, taglio delle frequenze, e ri-esportazione dell'immagine sono scritte in C (compilato come C++ per via della dipendenza da una libreria per la manipolazione di immagini da noi utilizzata, scritta in C++). Il codice che disegna e stampa a schermo l'istogramma delle frequenze è scritto in Python.

\subsection*{Lettura delle immagini}

Le immagini campione fornite su \texttt{k.matapp.unimib.it} non sono memorizzate come BMP pure e senza compressione (ovvero: il file non contiene semplicemente una matrice con i valori in \emph{greyscale} dei pixel), ma sono compresse usando una \emph{run-length encoding} a 8 bits. Per questo motivo, abbiamo scartato l'idea di scrivere "a mano" un lettore di files BMP, e deciso di utilizzare una libreria esterna (possibilmente il più leggera possibile) per l'importazione delle immagini.

Una libreria di semplice utilizzo che abbiamo trovato è CImg\footnote{cimg.sourceforge.net}, scritta in C++. Non richiede installazione, e consiste in un unico \emph{header file} \texttt{CImg.h} che va semplicemente incluso nella \emph{directory} del progetto. L'utilizzo di una libreria C++ ci ha costretti a compilare il resto del nostro codice (C99 puro) come C++, ma le differenze tra i due linguaggi sono minimali e gli adattamenti che abbiamo dovuto fare sul codice già scritto sono stati di poco conto.

Il codice (C++) utilizzato per la lettura dell'immagine è il seguente:

\begin{minted}[frame=single]{c++}
CImg<uint8_t> image(argv[1]);   // argv[1] contiene il nome del file
uint8_t * img = image.data();   // i dati rilevanti salvati in un array
int m = image.height();         // estraendo poi dall'header altezza
int n = image.width();          //         e larghezza dell'immagine
\end{minted}

Da notare che la matrice che contiene i pixel è di tipo \texttt{uint8\_t}, dato che la consegna del progetto richiedeva di operare esclusivamente su immagini \emph{greyscale}. CImg è in grado di leggere senza problemi immagini BMP compresse con RLE, e i dati ritornati dal metodo \texttt{data( )} rappresentano i pixel già scompattati.

\subsection*{DCT2}

Scrivendo in C99, è banale chiamare direttamente FFTW3. La libreria è facile da installare e ben documentata. La chiamata avviene in due passaggi: prima di tutto si crea un piano (\emph{plan}) con le caratteristiche della FT che si vuole operare sui dati, successivamente lo si esegue.

Le funzione da usarsi per la creazione di un \emph{plan} di DCT bidimensionale da $\mathbb{R}$ ad $\mathbb{R}$ è \texttt{fftw\_plan\_r2r\_2d( )}. Alla funzione, oltre che i parametri con i dati da trasformare, vanno passate due flag (una per ognuna delle due dimensioni) che specificano quale tipo di trasformata applicare. Nel nostro caso, per entrambe, si tratta di \texttt{FFTW\_REDFT10}:

FFTW\_REDFT10 (DCT-II, 'the' DCT):
\begin{quotation}
even around $j=-0.5$ and even around $j=n-0.5$
\end{quotation}

\begin{minted}[frame=single]{c}
double * dct2(int m, int n, uint8_t * array)
{
    // setup input e output
    double * out = (double *) malloc(sizeof(double) * m * n);
    double * in  = to_double(m, n, array);

    // do the fftw3 magic
    fftw_plan plan = fftw_plan_r2r_2d(m, n, in, out, FFTW_REDFT10,
				      FFTW_REDFT10, FFTW_ESTIMATE);
    fftw_execute(plan);

    // normalizzazione
    normalizza(m, n, out);
    return out;
}
\end{minted}

La funzione \texttt{dct2} prende un array di \texttt{uint8\_t} e restituisce un array di \texttt{double} contenente la DCT2 dell'input, normalizzata come richiesto dalla consegna per riflettere i risultati della funzione \texttt{dct2} di Matlab.

La funzione di normalizzazione è implementata come indicato nella consegna:

\begin{minted}[frame=single]{c}
void normalizza(int m, int n, double * array)
{
    double norm1 = 4 * sqrt(m/2.0) * sqrt(n/2.0);
    double norm2 = sqrt(2);
    
    for(int i = 0; i < m; i++){            // globale
        for(int j = 0; j < n; j++)
            array[i*n + j] /= norm1;     
    }

    for(int i = 0; i < n; i++)            // prima riga
        array[i] /= norm2;

    for(int j = 0; j < (m*n); j+=n)       // prima colonna
        array[j] /= norm2;
}
\end{minted}

\subsection*{Calcolo Istogrammi}
Per il calcolo degli istogrammi, una funzione \texttt{hist( )} accetta come parametro il numero di \emph{bins} e dopo averne calcolato la larghezza (basandosi sul valore massimo contenuto nella DCT), conteggia, per ognuno dei \emph{bins}, il numero di coefficienti presenti all'interno del \emph{bin}.

\begin{minted}[frame=single]{c}
void hist(int m, int n, int bins, double * array, const char * name)
{
    double max; // il valore massimo in array
    /* calcola max */
   
    double bins_width = (max + 100.0) / bins;
    int * hist = calloc(bins, sizeof(int));
    for(int i = 0; i < m; i++){
        for(int j = 0; j < n; j++)
	        hist[ (int)abs(array[i*n + j] / bins_width) ]++;
    }

    // continua ...
\end{minted}

\texttt{hist( )} salva su poi su file il risultato della computazione:
\begin{minted}[frame=single]{c}
    // ... continua
   
    FILE * fileptr = fopen(name, "w");
    fprintf(fileptr, "bins\t%d \n", bins);
    fprintf(fileptr, "width\t%.3f \n", bins_width);
    fprintf(fileptr, "max\t%.3f \n", max);
    for(int i = 0; i < bins; i++)
	fprintf(fileptr, "%d\n", hist[i]);

    fclose(fileptr);
}
\end{minted}


\subsection*{Taglio delle frequenze}

Dopo la visualizzazione dell'istogramma delle frequenze, l'utente può specificare un valore di soglia per il taglio della DCT2: ogni componente che sia in valore assoluto minore del valore di soglia è portata a zero.


\begin{minted}[frame=single]{c}
void cut_dct(int m, int n, double val, double * array)
{
    for(int i = 0; i < m; i++){
        for(int j = 0; j < n; j++){
            if(abs(array[i*n + j]) < val)
                array[i*n + j] = 0.0;
        }
    }
}
\end{minted}

La modifica è fatta, per ragioni di efficienza, \emph{in-place}: l'array originale viene perso e i valori originali sovrascritti in memoria.

\subsection*{IDCT2}

La funzione che opera il calcolo della IDCT2 è simile alla \texttt{dct2( )}. Per la creazione del piano è necessario passare come flag DCT \texttt{FFTW\_REDFT01} (che è l'inversa della \texttt{FFTW\_REDFT10} usata in precedenza per la DCT2), mentre la denormalizzazione divide per quello che la normalizzazione ha moltiplicato.

Dopo la denormalizzazione e il calcolo della IDCT2 (di cui non riportiamo il codice, essendo questo del tutto analogo a quello della \texttt{dct2( )} precedentemente descritta), è necessario applicare una trasformazione ulteriore. Infatti (citando dalla documentazione della FFTW3):
\begin{quotation}
the transforms computed by FFTW are unnormalized, exactly like the corresponding real and complex DFTs, so computing a transform followed by its inverse yields the original array scaled by $N$, where $N$ is the logical DFT size. For REDFT01, $N=2n$. 
\end{quotation}

Nel caso bidimensionale, il fattore di scala è $2n \cdot 2m = 4nm$, e va applicato dopo la denormalizzazione e  il calcolo della IDCT2.

\begin{minted}[frame=single]{c}
void scala(int m, int n, double * array)
{
    double scale = 4.0 * m * n;

    for(int i = 0; i < m; i++){
        for(int j = 0; j < n; j++)
            array[i*n + j] = array[i*n + j] / scale;
    }
}
\end{minted}

Senza questo passaggio, non è possibile confrontare direttamente la matrice \texttt{uint8\_t} di partenza e il risultato della IDCT2 (tralasciando in questo momento la questione taglio frequenze, che ovviamente va a modificare la matrice di partenza).

\subsection*{Esportazione delle immagini}
Per l'esportazione delle immagini tagliate utilizziamo ancora la CImg citata in precedenza. Il codice (C++):
\begin{minted}[frame=single]{c++}
CImg<uint8_t> image2(img2, n, m, 1, 1);
image2.save(strcat(argv[1], "cut.bmp"), -1, 3);;
\end{minted}

Nella prima riga viene costruito un oggetto CImg (\texttt{image2}) a partire dalla matrice dei pixel risultato della IDCT2 (\texttt{img2}). Nella seconda riga l'oggetto viene salvato su file con suffisso "cut" nel nome.

\section*{Risultati dei test}



\end{document}












